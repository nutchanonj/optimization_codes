\documentclass{article}
\usepackage[utf8]{inputenc}

\usepackage{amsmath}
\usepackage{graphicx}
\usepackage{indentfirst}

%--------Margin of the page------------%
\usepackage[margin=1in]{geometry}

%--------Code Snipplet Setting------------%
\usepackage{listings}
\usepackage{xcolor}

\definecolor{codegreen}{rgb}{0,0.6,0}
\definecolor{codegray}{rgb}{0.5,0.5,0.5}
\definecolor{codepurple}{rgb}{0.58,0,0.82}
\definecolor{backcolour}{rgb}{0.95,0.95,0.92}

\lstdefinestyle{mystyle}{
    backgroundcolor=\color{backcolour},   
    commentstyle=\color{codegreen},
    keywordstyle=\color{magenta},
    numberstyle=\tiny\color{codegray},
    stringstyle=\color{codepurple},
    basicstyle=\ttfamily\footnotesize,
    breakatwhitespace=false,         
    breaklines=true,                 
    captionpos=b,                    
    keepspaces=true,                 
    numbers=left,                    
    numbersep=5pt,                  
    showspaces=false,                
    showstringspaces=false,
    showtabs=false,                  
    tabsize=2
}

\lstset{style=mystyle}
%--------Code Snipplet Setting------------%

\begin{document}

\section*{Golden section search algorithm}

This example tries to find the minimizer of $f(x) = 10(x-1)^4 - 4\sin(3x)$

The code for \lstinline{golden.m} is written as follow:

\lstinputlisting[language=Matlab]{golden.m}

The code for function $f$ is written as follow:

\lstinputlisting[language=Matlab]{f.m}

This is the script used to test the function and printing the result. The easiest systematic way to find suitable $b$ is to increment its value until it works.

\lstinputlisting[language=Matlab]{script_test.m}

And the result is 

\lstinputlisting[]{result.txt}

The result up to 4 sig. fig. is 0.5969.

\end{document}